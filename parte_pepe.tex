\section{La empresa: Elementos y funciones}

La empresa es una organización que transforma un conjunto de recursos físicos, monetarios y cognitivos en bienes y/o servicios, con el objetivo principal de obtener beneficios \cite{}
Atendiendo a esta definición podemos efectivamente definir a Twitter como una empresa, y por tanto en los siguientes subapartados vamos a clasificarla atendiendo a varios criterios:

\subsection{Tamaño}

Según el tamaño tenemos que Twitter es una empresa grande ya que, según datos de 2015, tiene un total de 3.638 trabajadores y un total de ingresos de 2 mil millones de dólares.

\subsection{Sector}

Al tratarse de una empresa que ofrece servicios al público general, y ni extrae materias primas ni las transforma, podemos encasillarla en el sector terciario o sector servicios.

\subsection{Ámbito geográfico de actividad}

Su sede está en San Francisco (California), sin embargo opera en todo el mundo y ofrece gran variedad de idiomas para que cualquier persona del mundo pueda acceder a ella, por tanto estamos hablando de una empresa internacional.

\subsection{Forma jurídica}

Twitter tiene la estructura de Sociedad Anónima, es decir, está estructurada en acciones que cualquiera puede comprar e invertir en ella.

\subsection{Procedencia del capital}

Desde el año 2013 (5 años después de que la red se lanzase) Twitter es una empresa de capital abierto debido a la realización de una Oferta Pública Inicial, sin embargo es como tal una empresa privada ya que el capital es privado, pertenece a accionistas y no al estado.

\section{Dirección y gobierno de la empresa: funciones y niveles}

La función de dirección consiste en gestionar todo el funcionamiento de la empresa para conseguir alcanzar los objetivos empresariales. Para ello es imprescindible que los directivos, personas al cargo de cada tarea y con autoridad, tengan además las capacidades de motivación, liderazgo y comunicación.

En el caso de Twitter, Inc. la función de dirección comprende varios equipos con distintas funciones.

\subsection{Propiedad}

La estructura de propiedad se relaciona con el modo en que se distribuye el capital de las empresas entre sus propietarios legales (accionistas en este caso). La mayor parte de grupos de propiedad en Twitter son de capital industrial o empresarial, en concreto de capital riesgo, centradas en financiar a empresas con alto potencial de crecimiento, pero también hay gran presencia de particulares. Tenemos los siguientes accionistas principales:

\begin{itemize}

\item Kleiner Perkins Caufield and Bayers
\item Benchmark
\item Spark Capital
\item Insight Venture Partners
\item Union Square Ventures
\item Institutional Venture Partners
\item DST Global
\item Alwaleed Bin Talal

\end{itemize}

\subsection{Dirección ejecutiva}

El director ejecutivo de Twitter es Jack Dorsey, siendo también es co-fundador de esta. Además, Ed Ho y Kayvon Beykpour son gerentes generales.

La función principal de la dirección ejecutiva es actuar como cabeza visible de la empresa, informando sobre los objetivos, logros o participación de la empresa, así como gestionando la organización y los empleados.

\subsection{Equipo de Recursos Tecnológicos}



\section{Análisis DAFO}

