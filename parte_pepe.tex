\section{La empresa: Elementos y funciones}

La empresa es una organización que transforma un conjunto de recursos físicos, monetarios y cognitivos en bienes y/o servicios, con el objetivo principal de obtener beneficios.
Atendiendo a esta definición podemos efectivamente definir a Twitter, Inc. como una empresa, y por tanto en los siguientes subapartados vamos a clasificarla atendiendo a varios criterios:

\subsection{Tamaño}

Según el tamaño tenemos que Twitter, Inc. es una empresa grande ya que, según datos de 2017, tiene un total de 3.372 trabajadores y un total de ingresos de 7 mil millones y medio de dólares.

\subsection{Sector}

Al tratarse de una empresa que ofrece servicios al público general, y ni extrae materias primas ni las transforma, podemos encasillarla en el sector terciario o sector servicios.

\subsection{Ámbito geográfico de actividad}

Su sede está en San Francisco (California), sin embargo opera en todo el mundo y ofrece gran variedad de idiomas para que cualquier persona del mundo pueda acceder a ella, por tanto estamos hablando de una empresa internacional.

\subsection{Forma jurídica}

Twitter tiene la estructura de Sociedad Anónima, es decir, está estructurada en acciones que cualquiera puede comprar o invertir en ella.

\subsection{Procedencia del capital}

Desde el año 2013 (5 años después de que la red se lanzase) Twitter, Inc. es una empresa de capital abierto debido a la realización de una Oferta Pública Inicial, sin embargo es como tal una empresa privada ya que el capital es privado, pertenece a accionistas y no al estado.

\subsubsection{Propiedad}

La estructura de propiedad se relaciona con el modo en que se distribuye el capital de las empresas entre sus propietarios legales (accionistas en este caso). La mayor parte de grupos de propiedad en Twitter, Inc. son de capital industrial o empresarial, en concreto de capital riesgo, centradas en financiar a empresas con alto potencial de crecimiento, pero también hay gran presencia de particulares. Tenemos los siguientes accionistas principales:

\begin{itemize}

\item Kleiner Perkins Caufield and Bayers
\item Benchmark
\item Spark Capital
\item Insight Venture Partners
\item Union Square Ventures
\item Institutional Venture Partners
\item DST Global
\item Alwaleed Bin Talal

\end{itemize}

Por tanto como modelo de gobierno corporativo, Twitter, Inc. utiliza el modelo anglosajón, principalmente utilizado en EEUU y caracterizado por tener gran independencia y un número de acciones muy repartido entre varios grupos.

\section{Dirección y gobierno de la empresa: funciones y niveles}

La función de dirección consiste en gestionar todo el funcionamiento de la empresa para conseguir alcanzar los objetivos empresariales. Para ello es imprescindible que los directivos, personas al cargo de cada tarea y con autoridad, tengan además las capacidades de motivación, liderazgo y comunicación.

\subsection{Dirección ejecutiva}

Omid R. Kordestani es el presidente ejecutivo de Twitter desde 2015.

El director ejecutivo (CEO) de Twitter, Inc. es Jack Dorsey, siendo también co-fundador de esta. 
La función principal de la dirección ejecutiva es actuar como cabeza visible de la empresa, informando sobre los objetivos, logros o participación de la empresa, así como gestionando la organización y los empleados.

El resto del equipo de liderazgo consta de los siguientes cargos:

\begin{itemize}

\item Parag Agrawal - Director de Recursos Tecnológicos

Está a cargo de dirigir la estrategia empresarial en materia de tecnología y supervisar las áreas de aprendizaje automático e inteligencia artificial en los equipos de productos para usuarios finales, productos rentabilizables y ciencia. Desde su incorporación a Twitter en 2011, Parag ha dirigido las actividades de ampliación de los sistemas de anuncios y ha dado un nuevo impulso al crecimiento de la base de usuarios al mejorar la relevancia de la cronología de inicio.  

\item Ned Segal - Director ejecutivo de Finanzas

Tiene a su cargo la supervisión de las áreas de la empresa relacionadas con finanzas, contabilidad, desarrollo corporativo, seguridad corporativa, análisis y planificación financieros (FPA), relaciones con los inversores, propiedades inmobiliarias e instalaciones corporativas, auditorías internas, procesos tributarios y tesorería.

\item Leslie Berland - Directora de Markting y jefa de Recursos Humanos

Es responsable de las comunicaciones y el marketing de ventas, productos y consumidores a nivel mundial de Twitter. Como jefa de Recursos Humanos, Leslie lidera los equipos de Recursos Humanos, Contrataciones, y Desarrollo Organizacional y de Aprendizaje de Twitter.

\item Vijaya Gadde - Jefa de Asuntos Legales, Política y Seguridad

Maneja los temas de asuntos legales, políticas públicas, y seguridad y privacidad de Twitter.

\item Kayvon Beykpour - Gerente general, Vídeo

Encabeza los productos de video, cargo en el que supervisa los videos, el contenido en vivo y Periscope, una plataforma de videos en vivo que cofundó, donde las personas pueden crear, mirar, descubrir y compartir videos en vivo.

\item Ed Ho - Gerente general, Productos para el Consumidor e Ingeniería

Dirige al equipo responsable de desarrollar y operar el producto de Twitter.

\item Bruce Falck - Gerente general, Productos Rentabilizables e Ingeniería

Supervisa la estrategia e implementación de productos publicitarios para especialistas en marketing en Twitter.

\item Mat Derella - Vicepresidente de Ingresos y Asociaciones

Supervisa las organizaciones de ingresos publicitarios de la empresa, incluidas las ventas publicitarias globales, las asociaciones de contenido globales, el contenido en directo y las operaciones de generación de ingresos.

\item Grace Kim - Vicepresidenta de Investigación de Usuarios y Diseño

Dirige las iniciativas de investigación y diseño de productos de consumo y comerciales.

\end{itemize}

\subsection{Dirección de primera línea}

En el caso de Twitter, Inc. la función de dirección comprende además varios equipos de primera línea con distintas funciones. Podemos distinguir tres grupos diferenciados, cada uno con varios equipos: \textit{Build the product}, \textit{Keep us running} y \textit{Promote the business}

\section{Responsabilidad Social Corporativa}

Podemos definir la Responsabilidad Social Corporativa (RSC) como la integración voluntaria, por parte de las empresas, de las preocupaciones sociales y medioambientales en sus operaciones comerciales y en sus relaciones con sus interlocutores o partes interesadas (stakeholders); teniendo en cuenta también los aspectos económico-financieros. 

Con respecto a esto, Twitter, Inc. está bastante concienciada y realiza una gran cantidad de acciones, gracias a su conjunto de campañas denominadas como \textit{Twitter for Good}. Con el eslogan \textit{$"$Creemos que el intercambio libre de información puede tener un impacto positivo en el mundo.$"$} Twitter, Inc. se compromete a ofrecer seguridad, igualdad y libertad de expresión.

Realiza acciones de RSC en los siguientes ámbitos:

\begin{itemize}

\item \textbf{Seguridad y educación en Internet} apoyando a las organizaciones que se ocupan de los problemas relacionados con la seguridad en Internet, tales como la intimidación, el abuso, la violencia de género y las conductas de incitación al odio.

\item \textbf{Libertad de expresión y libertad civil} apoyando a las organizaciones que defienden y promueven los derechos y la libertad de expresión en Internet. Por ejemplo, participando en el \textit{Oslo Freedom Forum} (2016).

\item \textbf{Mujeres y minorías con baja representación en el sector de la tecnología} apoyando las actividades destinadas a que las mujeres y los miembros de minorías con baja representación de cualquier edad tengan acceso a programas en las áreas de ciencia, tecnología, ingeniería y matemáticas. Para fomentar el empoderamiento de las mujeres y su seguridad en Internet, se inició el hashtag \#PositionofStrength.

\item \textbf{Acceso y adopción universales}. Twitter cree que todas las comunidades deben tener acceso a Internet a un costo asequible. Por tanto, trabajan en estrecha colaboración con ONG y organizaciones benéficas de todo el mundo para que estas puedan dar a conocer su misión en Twitter y para exponer los beneficios del acceso a Internet a través del indispensable trabajo que estas ONG realizan en sus comunidades.

\item \textbf{Respuesta ante emergencias y recuperación ante desastres}. Cuando ocurren emergencias o desastres naturales, ofrecen herramientas y programas que facilitan la comunicación entre las víctimas, el personal de primera intervención y las actividades de ayuda humanitaria.

\item \textbf{Participación en la comunidad} colaborando con organizaciones sin fines de lucro de todo el mundo para brindarles asesoramiento sobre cómo desarrollar una presencia en Twitter más efectiva que fomente la interacción. En San Francisco, $‎@$NeighborNest‎ es un centro familiar de aprendizaje para la comunidad local situado justo enfrente de la sede de Twitter.

\item \textbf{Participación de los empleados}. Varias veces al año, los empleados participan en un día de servicio global llamado \#TwitterForGood Day, colaborando con organizaciones sin fines de lucro y pasando el día ayudándolas a servir a las comunidades locales de Twitter.

\end{itemize}

\section{Planificación}

La planificación trata de establecer un curso de acción por adelantado, previendo los recursos materiales y humanos necesarios para lograr los objetivos propuestos. Todo ello está dispuesto por la función de dirección anteriormente mencionada.

Las premisas de planificación son suposiciones de hechos y comportamientos que se producirán a lo largo del periodo considerado en la planificación. Se basan en un doble análisis: 

\begin{itemize}

\item \textbf{Análisis del ámbito externo:} amenazas y oportunidades
\item \textbf{Análisis del ámbito interno:} debilidades y fortalezas 

\end{itemize}

\subsection{Análisis DAFO}

\subsubsection{Debilidades}

\begin{itemize}

\item \textbf{Baja tasa de retención de usuarios:} Más del 40\% de usuarios en Twitter la acaban abandonando rápidamente, lo que muestra la poca capacidad de la empresa en mantenerlos interesados en su producto.
\item \textbf{Distribución desigual de tweets:} El 90\% de los tweets viene sólo del 10\% de los usuarios de la plataforma, aunque pese a eso se siga manteniendo muy fuerte ya que este volumen de tweets es muy amplio.
\item \textbf{No hay nuevas funciones desde su creación.}
\item \textbf{No hay una fuente de ingresos sólida:} Ya que es gratis y sin anuncios, hay una gran incertidumbre sobre qué será de los ingresos a largo plazo.

\end{itemize}

\subsubsection{Amenazas}

\begin{itemize}

\item \textbf{Modelo de no beneficio después de 10 años:} Después de una década aún no han encontrado una forma que otorgue beneficios, por lo que al largo plazo puede resultar mortal.
\item \textbf{Cuentas falsas e incidentes de hackeo:} Si los datos son falsos, o se hackean de alguna manera, no les sirven de nada a Twitter y resultan un problema.
\item \textbf{Competidores como \textit{Facebook}:} Pese a Twitter estar bien diferenciada de esta, \textit{Facebook} no deja de innovar su plataforma, por lo que puede hacer de la existencia de Twitter algo vulnerable.
\item \textbf{Nuevas redes sociales parecidas que realmente innovan}: Pueden suponer la pérdida de terreno de Twitter, al estar esta tan estática.

\end{itemize}

\subsubsection{Fortalezas}

\begin{itemize}

\item \textbf{Buena imagen de marca para cuentas creíbles de famosos:} Twitter ha conseguido que el público general no sea escéptico a la hora de ver la cuenta de un famoso, y además tener herramientas para conocer que efectivamente esa cuenta es real (tick de verificación).
\item \textbf{Pioneros en redes sociales:} Twitter entró al mercado de las redes sociales en 2006, cuando aún no había apenas alternativas, por lo que gozan de ser de las primeras en este campo.
\item \textbf{Gran tecnología} que ha supuesto no tener apenas nunca fallos técnicos, y por tanto hacer que la gente confíe en él. 
\item \textbf{Es el sitio preferido por empresas y famosos para comunicarse.}
\item \textbf{Nombre y logo muy reconocible:} El pájaro azul de Twitter es su seña de identidad más notable, y saben usarla perfectamente para atraer clientes.
\item \textbf{Gran canal de comunicación} debido a su carácter de red social en tiempo real.
\item \textbf{Gran integración con otras plataformas} pudiendo compartir links de otros sitios como \textit{Youtube} o \textit{Blogger}, entre otras, así como pudiendo ser facil insertar links de Twitter en cualquier página web.
\item \textbf{Cantidad de datos:} Twitter tiene en su base de datos a más de 300 millones de personas, convirtiéndose esto en su fortaleza principal.
\item \textbf{Simple de entender y gratis.}

\end{itemize}

\subsubsection{Oportunidades}

\begin{itemize}

\item \textbf{Servicios de atención al consumidor para empresas:} Podrían ofrecer servicios a empresas para que gestionen de mejor manera su publicidad y atención al cliente.
\item \textbf{Integrarse con aún más plataformas.}
\item \textbf{Aumentar la utilidad de los Momentos:} Los Momentos son una parte de Twitter que trata noticias y organiza tweets por temas, entre otras cosas, y podría dársele más protagonismo ya que ahora parece una parte ajena a la propia red social y no está integrada en la propia página principal.
\item \textbf{Innovación:} Twitter está muy estancada en su diseño y uso actual, y podría incluir novedades para atraer a más gente.
\item \textbf{Buscadores web:} Twitter, al tener tal cantidad de información, podría entrar en el negocio de los buscadores web con gran éxito.

\end{itemize}

\subsection{Misión}

"Hacer del mundo un lugar mejor aprovechando el poder positivo de Twitter a través de la participación cívica, las tareas de voluntariado y las asociaciones con organizaciones benéficas de todo el mundo. Por tanto, nos comprometemos con esta misión concentrando su ayuda allí donde puedan tener el mayor impacto. "

\subsection{Visión}

"Llegar al mayor número posible de personas, conectándolas todas al mundo, mediante nuestras capacidades de compartir información y nuestros plataformas para distribuir los productos, y además ser una de las empresas de Internet con mayor generación de ingresos del mundo"

\section{Objetivos}

Los objetivos especifican las situaciones futuras que el gerente espera lograr. Son, en definitiva, fines concretos, normalmente cuantificables y con un horizonte temporal delimitado, a los que se dirige la actividad de una organización, y que en este caso podemos resumir en su misión y visión anteriormente mencionadas, que como podemos ver son la parte esencial de la red social.

\section{Control}

El control consiste en verificar si todo se realiza conforme al programa adoptado, a las órdenes impartidas y a los principios admitidos. Tiene la finalidad de señalar las faltas y los errores, a fin de que se pueda repararlos y evitar su repetición. Se aplica a todo, a las cosas, a las personas y a los actos. 

En el caso de Twitter, Inc. la función de control tiene poca presencia. La empresa se caracteriza por estar muy descentralizada, teniendo cada departamento que realizar su trabajo y tomar sus decisiones, y también se caracteriza por no prestar demasiada atención a la innovación. Por lo que, en líneas generales, en Twitter se lleva trabajando durante 10 años con objetivos muy parecidos y hay pocos casos en los que la función de control sale a realizar su función.