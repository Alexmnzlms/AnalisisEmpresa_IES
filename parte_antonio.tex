
\section{La organización en Twitter, Inc.}

La función de organización se basa, principalmente, en dividir el trabajo, tanto humano como material, para posteriormente coordinar las distintas tareas de forma agrupada para la correcta ejecución de los planes establecidos. Para esta organización debemos tener en cuenta distintos factores, como la misión, los objetivos de la empresa, el uso de herramientas para la construcción de esta estructura, entre otros.

En la empresa que analizamos, Twitter, Inc., ya hemos mencionado anteriormente algunos de estos (como la misión).

\subsection{Principios organizativos fundamentales.}

\subsubsection{Principio de la división del trabajo.}

Siguiendo estos principio, Twitter, Inc., se distribuye en distintos equipos de trabajo, entre los que se encuentran:

\textbf{Construir el producto}

El grupo \textit{Build the product} comprende los siguientes equipos:

\begin{itemize}

\item \textit{Data Science and Analytics}
\item \textit{Infraestructure Engineering}
\item \textit{Product}
\item \textit{Software Engineering}
\item \textit{User Services}
\item \textit{Design and Research}

\end{itemize}

\textbf{Grupo: Seguir manteniéndonos}

El grupo \textit{Keep us running} comprende los siguientes equipos:

\begin{itemize}

\item \textit{Finance}
\item \textit{Legal and Public Policy}
\item \textit{People}
\item \textit{Workplace}

\end{itemize}

\textbf{Grupo: Promover el negocio}

El grupo \textit{Promote the business} comprende los siguientes equipos:

\begin{itemize}

\item \textit{Marketing and Communications}
\item \textit{Sales and Partnerships}

\end{itemize}

\textbf{Grupo: Nuestras marcas}

El grupo \textit{Our family of brands} comprende los siguientes equipos:

\begin{itemize}

\item \textit{Periscope}
\item \textit{MoPub}

\end{itemize}


Todos estos equipos se distribuyen entre 35 países alrededor del mundo.

\subsubsection{Principio de la especialización del trabajo.}

Como vemos en el apartado anterior, existen grandes distinciones entre los distintos equipos de trabajo, por lo que es necesaria una especialización, que dependerá de cada equipo, encontrando así un alto nivel de especialización en cada uno de ellos.

\subsubsection{Principio de jerarquía, unidad de mando y ámbito de control.}

Estos principios, explicados y revisados en apartados anteriores, suponen una correcta organización del trabajo, esencial para el desarrollo de los objetivos de la empresa.

\subsubsection{Principio de descentralización.}

Como hemos visto, en Twitter, Inc. vemos una clara descentralización con los distintos grupos de trabajo, sin obviar las decisiones tomadas por los altos directivos.